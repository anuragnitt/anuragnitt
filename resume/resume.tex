%%%%%%%%%%%%%%%%%%%%%%%%%%%%%%%%%%%%%%%
% Deedy - One Page Two Column Resume
% LaTeX Template
% Version 1.2 (16/9/2014)
%
% Original author:
% Debarghya Das (http://debarghyadas.com)
%
% Original repository:
% https://github.com/deedydas/Deedy-Resume
%
% IMPORTANT: THIS TEMPLATE NEEDS TO BE COMPILED WITH XeLaTeX
%
% This template uses several fonts not included with Windows/Linux by
% default. If you get compilation errors saying a font is missing, find the line
% on which the font is used and either change it to a font included with your
% operating system or comment the line out to use the default font.
% 
%%%%%%%%%%%%%%%%%%%%%%%%%%%%%%%%%%%%%%
% 
% TODO:
% 1. Integrate biber/bibtex for article citation under publications.
% 2. Figure out a smoother way for the document to flow onto the next page.
% 3. Add styling information for a "Projects/Hacks" section.
% 4. Add location/address information
% 5. Merge OpenFont and MacFonts as a single sty with options.
% 
%%%%%%%%%%%%%%%%%%%%%%%%%%%%%%%%%%%%%%
%
% CHANGELOG:
% v1.1:
% 1. Fixed several compilation bugs with \renewcommand
% 2. Got Open-source fonts (Windows/Linux support)
% 3. Added Last Updated
% 4. Move Title styling into .sty
% 5. Commented .sty file.
%
%%%%%%%%%%%%%%%%%%%%%%%%%%%%%%%%%%%%%%%
%
% Known Issues:
% 1. Overflows onto second page if any column's contents are more than the
% vertical limit
% 2. Hacky space on the first bullet point on the second column.
%
%%%%%%%%%%%%%%%%%%%%%%%%%%%%%%%%%%%%%%


\documentclass[]{deedy-resume-openfont}
\usepackage{fancyhdr}
 
\pagestyle{fancy}
\fancyhf{}
 
\begin{document}

%%%%%%%%%%%%%%%%%%%%%%%%%%%%%%%%%%%%%%
%
%     TITLE NAME
%
%%%%%%%%%%%%%%%%%%%%%%%%%%%%%%%%%%%%%%
\namesection{}{Anurag Goyal}
{ \urlstyle{same}\href{mailto:anuraggoyal.awr@gmail.com}{anuraggoyal.awr@gmail.com} | +91 63751 73313
}

%%%%%%%%%%%%%%%%%%%%%%%%%%%%%%%%%%%%%%
%
%     COLUMN ONE
%
%%%%%%%%%%%%%%%%%%%%%%%%%%%%%%%%%%%%%%

\begin{minipage}[t]{0.33\textwidth} 

%%%%%%%%%%%%%%%%%%%%%%%%%%%%%%%%%%%%%%
%     EDUCATION
%%%%%%%%%%%%%%%%%%%%%%%%%%%%%%%%%%%%%%

\section{Education} 

\subsection{NIT - Tiruchirappalli}
\location{Tiruchirappalli, Tamil Nadu}
\descript{B.Tech in Computer Science \\
and Engineering}
\location{Expected May 2023}
\location{Cum. GPA: 8.16}
\sectionsep

\subsection{S.R. Public Sr. Sec. School}
\location{Kota, Rajasthan}
\descript{Science with Physical Education}
\location{CBSE Std XII : 89.2\%}
\sectionsep

\subsection{St. Anselms Sr. Sec. School}
\location{Alwar, Rajasthan}
\location{CBSE Std X : 9.8 CGPA}
\sectionsep

%%%%%%%%%%%%%%%%%%%%%%%%%%%%%%%%%%%%%%
%     LINKS
%%%%%%%%%%%%%%%%%%%%%%%%%%%%%%%%%%%%%%

\section{Links} 
Github:// \href{https://github.com/anuragnitt}{\bf anuragnitt} \\
LinkedIn://  \href{https://www.linkedin.com/in/anurag-goyal-b5884a1ab/}{\bf anurag-goyal}
\sectionsep

%%%%%%%%%%%%%%%%%%%%%%%%%%%%%%%%%%%%%%
%     SKILLS
%%%%%%%%%%%%%%%%%%%%%%%%%%%%%%%%%%%%%%

\section{Skills}
\subsection{Programming}
\location{LANGUAGES}
C++ \textbullet{} Python \textbullet{} JavaScript \\
UNIX Shell \textbullet{} PHP \textbullet{} C \\
\location{FRAMEWORKS AND LIBRARIES}
Pycryptodome \textbullet{} Flask \\
\location{TOOLS}
Git \textbullet{} Docker \textbullet{} MySQL \textbullet{} Apache
\sectionsep

%%%%%%%%%%%%%%%%%%%%%%%%%%%%%%%%%%%%%%
%     CERTIFICATIONS
%%%%%%%%%%%%%%%%%%%%%%%%%%%%%%%%%%%%%%

\section{Certifications}
\subsection{Google Cloud Platform}
A Google certified practical course on
Deployment, Networking and Data Analysis
on Google Cloud Platform.\newline
\location{\href{https://www.qwiklabs.com/public_profiles/2e1928cc-a44e-4bcc-9b54-0de68d9b14d7}{GCP Skill Badges}}
\location{\href{https://drive.google.com/file/d/1YGuBYzLVNAck5-LRyPsmowuu5hCkXTmN/view?usp=sharing}{Certificate}}
\sectionsep

%%%%%%%%%%%%%%%%%%%%%%%%%%%%%%%%%%%%%%
%     SOCIETIES
%%%%%%%%%%%%%%%%%%%%%%%%%%%%%%%%%%%%%%

\section{Societies}
\subsection{Member At}
\location{\href{https://delta.nitt.edu}{DELTA FORCE}}
The official web team and \\
programming club of NIT Trichy.\newline
\subsection{Member At National \\
Service Scheme}
Organises social service based \\
activities in and around \\
Trichy, Tamil Nadu.
\sectionsep

%%%%%%%%%%%%%%%%%%%%%%%%%%%%%%%%%%%%%%
%
%     COLUMN TWO
%
%%%%%%%%%%%%%%%%%%%%%%%%%%%%%%%%%%%%%%

\end{minipage} 
\hfill
\begin{minipage}[t]{0.66\textwidth} 

%%%%%%%%%%%%%%%%%%%%%%%%%%%%%%%%%%%%%%
%     PROJECTS
%%%%%%%%%%%%%%%%%%%%%%%%%%%%%%%%%%%%%%

\section{Projects}
\runsubsection{Pragyan Premiere League}
\descript{| Private | Present}
\vspace{\topsep}
\begin{tightemize}
\item An annual virtual cricket league in which matches are algorithmically simulated.
\item Part of the team which worked on adding new features in the backend written in Golang.
\item Part of the three person team which hosted and deployed the project through Docker engine.
\item The app is annually launched to over 1400 registrants and played by users from 74 countries.
\end{tightemize}
\sectionsep

\runsubsection{Pragyan Capture The Flag}
\descript{| \href{https://github.com/anuragnitt/PragyanCTF-21}{Project Link} | Present}
\begin{tightemize}
\item An annual online jeopardy style cybersecurity event in which players attempt to exploit and attack an application or algorithm to retrieve the secret flag.
\item Part of the two person team which designed cryptography challenges comprising of various cryptography techniques.
\item The event is annually played by participants from many countries.
\end{tightemize}
\sectionsep

\runsubsection{Trash Track}
\descript{| \href{https://github.com/anuragnitt/trashTrack}{Project Link} | January 2021}
\begin{tightemize}
\item A realtime application which does geo-tracking of littered garbage, triggering alerts and mapping areas with high waste index.
\item Built at Pragyan Hackathon 2021.
\item Developed the backend, API (Flask) and integrated it with MySQL database.
\end{tightemize}
\sectionsep

\runsubsection{CLI based E-Mail client}
\descript{| \href{https://github.com/anuragnitt/Mail}{Project Link} | August 2020}
\begin{tightemize}
\item A CLI based E-Mail client developed in Python.
\item Application allows user to clone their GMail inbox to their device and send encrypted E-Mails with optional multimedia file attachments.
\end{tightemize}
\sectionsep

\runsubsection{Crypto-Library}
\descript{| \href{https://github.com/anuragnitt/Crypto-Library}{Project Link} | August 2020}
\begin{tightemize}
\item Implementations of DES, Triple DES and AES modes of encryption in Python.
\end{tightemize}
\sectionsep

\runsubsection{Programmer Lyf Bot}
\descript{| Private | January 2021}
\begin{tightemize}
\item A discord bot application which fetches informative and humorous content from Reddit and other popular public forums.
\item Designed a module which uses Reddit's API to fetch posts whose frequency varied based on its category.
\item The module is capable of separating text, multimedia and hyperlinks.
\end{tightemize}
\sectionsep

%%%%%%%%%%%%%%%%%%%%%%%%%%%%%%%%%%%%%%
%     ACHIEVEMENTS
%%%%%%%%%%%%%%%%%%%%%%%%%%%%%%%%%%%%%%

\section{Achievements} 
\begin{tabular}{rll}
2021         & Pragyan Hackathon Top 10 Finalists \\
2019	     & JEE-Main AIR 1536 \\
2019	     & JEE-Advanced AIR 5620 \\
2015	     & NSTSE Gold Medalist (State Level)\\
\end{tabular}
\sectionsep

\end{minipage} 
\end{document}  \documentclass[]{article}
